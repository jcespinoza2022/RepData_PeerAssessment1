% Options for packages loaded elsewhere
\PassOptionsToPackage{unicode}{hyperref}
\PassOptionsToPackage{hyphens}{url}
%
\documentclass[
]{article}
\author{}
\date{\vspace{-2.5em}}

\usepackage{amsmath,amssymb}
\usepackage{lmodern}
\usepackage{iftex}
\ifPDFTeX
  \usepackage[T1]{fontenc}
  \usepackage[utf8]{inputenc}
  \usepackage{textcomp} % provide euro and other symbols
\else % if luatex or xetex
  \usepackage{unicode-math}
  \defaultfontfeatures{Scale=MatchLowercase}
  \defaultfontfeatures[\rmfamily]{Ligatures=TeX,Scale=1}
\fi
% Use upquote if available, for straight quotes in verbatim environments
\IfFileExists{upquote.sty}{\usepackage{upquote}}{}
\IfFileExists{microtype.sty}{% use microtype if available
  \usepackage[]{microtype}
  \UseMicrotypeSet[protrusion]{basicmath} % disable protrusion for tt fonts
}{}
\makeatletter
\@ifundefined{KOMAClassName}{% if non-KOMA class
  \IfFileExists{parskip.sty}{%
    \usepackage{parskip}
  }{% else
    \setlength{\parindent}{0pt}
    \setlength{\parskip}{6pt plus 2pt minus 1pt}}
}{% if KOMA class
  \KOMAoptions{parskip=half}}
\makeatother
\usepackage{xcolor}
\IfFileExists{xurl.sty}{\usepackage{xurl}}{} % add URL line breaks if available
\IfFileExists{bookmark.sty}{\usepackage{bookmark}}{\usepackage{hyperref}}
\hypersetup{
  hidelinks,
  pdfcreator={LaTeX via pandoc}}
\urlstyle{same} % disable monospaced font for URLs
\usepackage[margin=1in]{geometry}
\usepackage{color}
\usepackage{fancyvrb}
\newcommand{\VerbBar}{|}
\newcommand{\VERB}{\Verb[commandchars=\\\{\}]}
\DefineVerbatimEnvironment{Highlighting}{Verbatim}{commandchars=\\\{\}}
% Add ',fontsize=\small' for more characters per line
\usepackage{framed}
\definecolor{shadecolor}{RGB}{248,248,248}
\newenvironment{Shaded}{\begin{snugshade}}{\end{snugshade}}
\newcommand{\AlertTok}[1]{\textcolor[rgb]{0.94,0.16,0.16}{#1}}
\newcommand{\AnnotationTok}[1]{\textcolor[rgb]{0.56,0.35,0.01}{\textbf{\textit{#1}}}}
\newcommand{\AttributeTok}[1]{\textcolor[rgb]{0.77,0.63,0.00}{#1}}
\newcommand{\BaseNTok}[1]{\textcolor[rgb]{0.00,0.00,0.81}{#1}}
\newcommand{\BuiltInTok}[1]{#1}
\newcommand{\CharTok}[1]{\textcolor[rgb]{0.31,0.60,0.02}{#1}}
\newcommand{\CommentTok}[1]{\textcolor[rgb]{0.56,0.35,0.01}{\textit{#1}}}
\newcommand{\CommentVarTok}[1]{\textcolor[rgb]{0.56,0.35,0.01}{\textbf{\textit{#1}}}}
\newcommand{\ConstantTok}[1]{\textcolor[rgb]{0.00,0.00,0.00}{#1}}
\newcommand{\ControlFlowTok}[1]{\textcolor[rgb]{0.13,0.29,0.53}{\textbf{#1}}}
\newcommand{\DataTypeTok}[1]{\textcolor[rgb]{0.13,0.29,0.53}{#1}}
\newcommand{\DecValTok}[1]{\textcolor[rgb]{0.00,0.00,0.81}{#1}}
\newcommand{\DocumentationTok}[1]{\textcolor[rgb]{0.56,0.35,0.01}{\textbf{\textit{#1}}}}
\newcommand{\ErrorTok}[1]{\textcolor[rgb]{0.64,0.00,0.00}{\textbf{#1}}}
\newcommand{\ExtensionTok}[1]{#1}
\newcommand{\FloatTok}[1]{\textcolor[rgb]{0.00,0.00,0.81}{#1}}
\newcommand{\FunctionTok}[1]{\textcolor[rgb]{0.00,0.00,0.00}{#1}}
\newcommand{\ImportTok}[1]{#1}
\newcommand{\InformationTok}[1]{\textcolor[rgb]{0.56,0.35,0.01}{\textbf{\textit{#1}}}}
\newcommand{\KeywordTok}[1]{\textcolor[rgb]{0.13,0.29,0.53}{\textbf{#1}}}
\newcommand{\NormalTok}[1]{#1}
\newcommand{\OperatorTok}[1]{\textcolor[rgb]{0.81,0.36,0.00}{\textbf{#1}}}
\newcommand{\OtherTok}[1]{\textcolor[rgb]{0.56,0.35,0.01}{#1}}
\newcommand{\PreprocessorTok}[1]{\textcolor[rgb]{0.56,0.35,0.01}{\textit{#1}}}
\newcommand{\RegionMarkerTok}[1]{#1}
\newcommand{\SpecialCharTok}[1]{\textcolor[rgb]{0.00,0.00,0.00}{#1}}
\newcommand{\SpecialStringTok}[1]{\textcolor[rgb]{0.31,0.60,0.02}{#1}}
\newcommand{\StringTok}[1]{\textcolor[rgb]{0.31,0.60,0.02}{#1}}
\newcommand{\VariableTok}[1]{\textcolor[rgb]{0.00,0.00,0.00}{#1}}
\newcommand{\VerbatimStringTok}[1]{\textcolor[rgb]{0.31,0.60,0.02}{#1}}
\newcommand{\WarningTok}[1]{\textcolor[rgb]{0.56,0.35,0.01}{\textbf{\textit{#1}}}}
\usepackage{longtable,booktabs,array}
\usepackage{calc} % for calculating minipage widths
% Correct order of tables after \paragraph or \subparagraph
\usepackage{etoolbox}
\makeatletter
\patchcmd\longtable{\par}{\if@noskipsec\mbox{}\fi\par}{}{}
\makeatother
% Allow footnotes in longtable head/foot
\IfFileExists{footnotehyper.sty}{\usepackage{footnotehyper}}{\usepackage{footnote}}
\makesavenoteenv{longtable}
\usepackage{graphicx}
\makeatletter
\def\maxwidth{\ifdim\Gin@nat@width>\linewidth\linewidth\else\Gin@nat@width\fi}
\def\maxheight{\ifdim\Gin@nat@height>\textheight\textheight\else\Gin@nat@height\fi}
\makeatother
% Scale images if necessary, so that they will not overflow the page
% margins by default, and it is still possible to overwrite the defaults
% using explicit options in \includegraphics[width, height, ...]{}
\setkeys{Gin}{width=\maxwidth,height=\maxheight,keepaspectratio}
% Set default figure placement to htbp
\makeatletter
\def\fps@figure{htbp}
\makeatother
\setlength{\emergencystretch}{3em} % prevent overfull lines
\providecommand{\tightlist}{%
  \setlength{\itemsep}{0pt}\setlength{\parskip}{0pt}}
\setcounter{secnumdepth}{-\maxdimen} % remove section numbering
\ifLuaTeX
  \usepackage{selnolig}  % disable illegal ligatures
\fi

\begin{document}

\begin{longtable}[]{@{}l@{}}
\toprule
\endhead
title: ``Reproducible Research Project 1'' \\
author: ``Jaime Espinoza'' \\
date: ``23/05/2022'' \\
output: \\
md\_document: \\
variant: markdown\_github \\
\bottomrule
\end{longtable}

github repo for rest of specialization:
\href{https://github.com/jcespinoza2022/datasciencecoursera}{Data
Science Coursera}

\hypertarget{introduction}{%
\subsection{Introduction}\label{introduction}}

Hoy en día es posible recopilar una gran cantidad de datos sobre el
movimiento personal mediante dispositivos de monitorización de la
actividad como Fitbit, Nike Fuelband o Jawbone Up. Este tipo de
dispositivos forman parte del movimiento del ``yo cuantificado'', un
grupo de entusiastas que toman mediciones sobre sí mismos con
regularidad para mejorar su salud, para encontrar patrones en su
comportamiento o porque son frikis de la tecnología. Pero estos datos
siguen estando infrautilizados, tanto porque los datos brutos son
difíciles de obtener como por la falta de métodos estadísticos y de
software para procesar e interpretar los datos.

Esta tarea utiliza los datos de un dispositivo de monitorización de la
actividad personal. Este dispositivo recoge datos a intervalos de 5
minutos a lo largo del día. Los datos consisten en dos meses de datos de
un individuo anónimo recogidos durante los meses de octubre y noviembre
de 2012 e incluyen el número de pasos dados en intervalos de 5 minutos
cada día.

Los datos para esta tarea pueden descargarse del sitio web del curso:

Dataset:
\href{https://d396qusza40orc.cloudfront.net/repdata\%2Fdata\%2Factivity.zip}{Activity
monitoring data}

Las variables incluidas en este conjunto de datos son:

pasos : Número de pasos dados en un intervalo de 5 minutos (los valores
faltantes se codifican como\color{rojo}{\verbo|NA|}N / A) date : La
fecha en la que se tomó la medida en formato AAAA-MM-DD intervalo :
Identificador del intervalo de 5 minutos en el que se tomó la medición
El conjunto de datos se almacena en un archivo de valores separados por
comas (CSV) y hay un total de 17 568 observaciones en este conjunto de
datos.

\hypertarget{carga-y-preprocesamiento-de-los-datos}{%
\subsection{Carga y preprocesamiento de los
datos}\label{carga-y-preprocesamiento-de-los-datos}}

Descomprimir los datos para obtener un archivo csv.

\begin{Shaded}
\begin{Highlighting}[]
\FunctionTok{library}\NormalTok{(}\StringTok{"data.table"}\NormalTok{)}
\FunctionTok{library}\NormalTok{(ggplot2)}

\NormalTok{fileUrl }\OtherTok{\textless{}{-}} \StringTok{"https://d396qusza40orc.cloudfront.net/repdata\%2Fdata\%2Factivity.zip"}
\FunctionTok{download.file}\NormalTok{(fileUrl, }\AttributeTok{destfile =} \FunctionTok{paste0}\NormalTok{(}\FunctionTok{getwd}\NormalTok{(), }\StringTok{\textquotesingle{}/repdata\%2Fdata\%2Factivity.zip\textquotesingle{}}\NormalTok{), }\AttributeTok{method =} \StringTok{"curl"}\NormalTok{)}
\FunctionTok{unzip}\NormalTok{(}\StringTok{"repdata\%2Fdata\%2Factivity.zip"}\NormalTok{,}\AttributeTok{exdir =} \StringTok{"data"}\NormalTok{)}
\end{Highlighting}
\end{Shaded}

\#\#Lectura de datos csv en Data.Table

\begin{Shaded}
\begin{Highlighting}[]
\NormalTok{activityDT }\OtherTok{\textless{}{-}}\NormalTok{ data.table}\SpecialCharTok{::}\FunctionTok{fread}\NormalTok{(}\AttributeTok{input =} \StringTok{"data/activity.csv"}\NormalTok{)}
\end{Highlighting}
\end{Shaded}

\hypertarget{cuuxe1l-es-la-media-del-nuxfamero-total-de-pasos-dados-al-duxeda}{%
\subsection{¿Cuál es la media del número total de pasos dados al
día?}\label{cuuxe1l-es-la-media-del-nuxfamero-total-de-pasos-dados-al-duxeda}}

\begin{enumerate}
\def\labelenumi{\arabic{enumi}.}
\tightlist
\item
  Calculate the total number of steps taken per day
\end{enumerate}

\begin{Shaded}
\begin{Highlighting}[]
\NormalTok{Total\_Steps }\OtherTok{\textless{}{-}}\NormalTok{ activityDT[, }\FunctionTok{c}\NormalTok{(}\FunctionTok{lapply}\NormalTok{(.SD, sum, }\AttributeTok{na.rm =} \ConstantTok{FALSE}\NormalTok{)), .SDcols }\OtherTok{=} \FunctionTok{c}\NormalTok{(}\StringTok{"steps"}\NormalTok{), by }\OtherTok{=}\NormalTok{ .(date)] }

\FunctionTok{head}\NormalTok{(Total\_Steps, }\DecValTok{10}\NormalTok{)}
\end{Highlighting}
\end{Shaded}

\begin{verbatim}
##           date steps
##  1: 2012-10-01    NA
##  2: 2012-10-02   126
##  3: 2012-10-03 11352
##  4: 2012-10-04 12116
##  5: 2012-10-05 13294
##  6: 2012-10-06 15420
##  7: 2012-10-07 11015
##  8: 2012-10-08    NA
##  9: 2012-10-09 12811
## 10: 2012-10-10  9900
\end{verbatim}

\begin{enumerate}
\def\labelenumi{\arabic{enumi}.}
\setcounter{enumi}{1}
\tightlist
\item
  Haz un histograma del número total de pasos dados cada día.
\end{enumerate}

\begin{Shaded}
\begin{Highlighting}[]
\FunctionTok{ggplot}\NormalTok{(Total\_Steps, }\FunctionTok{aes}\NormalTok{(}\AttributeTok{x =}\NormalTok{ steps)) }\SpecialCharTok{+}
    \FunctionTok{geom\_histogram}\NormalTok{(}\AttributeTok{fill =} \StringTok{"blue"}\NormalTok{, }\AttributeTok{binwidth =} \DecValTok{1000}\NormalTok{) }\SpecialCharTok{+}
    \FunctionTok{labs}\NormalTok{(}\AttributeTok{title =} \StringTok{"Daily Steps"}\NormalTok{, }\AttributeTok{x =} \StringTok{"Steps"}\NormalTok{, }\AttributeTok{y =} \StringTok{"Frequency"}\NormalTok{)}
\end{Highlighting}
\end{Shaded}

\begin{verbatim}
## Warning: Removed 8 rows containing non-finite values (stat_bin).
\end{verbatim}

\includegraphics{PA1_template_files/figure-latex/unnamed-chunk-4-1.pdf}

\begin{enumerate}
\def\labelenumi{\arabic{enumi}.}
\setcounter{enumi}{2}
\tightlist
\item
  Calcular y comunicar la media y la mediana del número total de pasos
  dados al día
\end{enumerate}

\begin{Shaded}
\begin{Highlighting}[]
\NormalTok{Total\_Steps[, .(}\AttributeTok{Mean\_Steps =} \FunctionTok{mean}\NormalTok{(steps, }\AttributeTok{na.rm =} \ConstantTok{TRUE}\NormalTok{), }\AttributeTok{Median\_Steps =} \FunctionTok{median}\NormalTok{(steps, }\AttributeTok{na.rm =} \ConstantTok{TRUE}\NormalTok{))]}
\end{Highlighting}
\end{Shaded}

\begin{verbatim}
##    Mean_Steps Median_Steps
## 1:   10766.19        10765
\end{verbatim}

\#\#¿Cuál es el patrón medio de actividad diaria?

\begin{enumerate}
\def\labelenumi{\arabic{enumi}.}
\tightlist
\item
  Realiza un gráfico de series temporales 𝚝𝚢𝚙𝚎 = ``𝚕'' del intervalo de
  5 minutos (eje x) y el número medio de pasos dados, promediado en
  todos los días (eje y)
\end{enumerate}

\begin{Shaded}
\begin{Highlighting}[]
\NormalTok{IntervalDT }\OtherTok{\textless{}{-}}\NormalTok{ activityDT[, }\FunctionTok{c}\NormalTok{(}\FunctionTok{lapply}\NormalTok{(.SD, mean, }\AttributeTok{na.rm =} \ConstantTok{TRUE}\NormalTok{)), .SDcols }\OtherTok{=} \FunctionTok{c}\NormalTok{(}\StringTok{"steps"}\NormalTok{), by }\OtherTok{=}\NormalTok{ .(interval)] }

\FunctionTok{ggplot}\NormalTok{(IntervalDT, }\FunctionTok{aes}\NormalTok{(}\AttributeTok{x =}\NormalTok{ interval , }\AttributeTok{y =}\NormalTok{ steps)) }\SpecialCharTok{+} \FunctionTok{geom\_line}\NormalTok{(}\AttributeTok{color=}\StringTok{"blue"}\NormalTok{, }\AttributeTok{size=}\DecValTok{1}\NormalTok{) }\SpecialCharTok{+} \FunctionTok{labs}\NormalTok{(}\AttributeTok{title =} \StringTok{"Avg. Daily Steps"}\NormalTok{, }\AttributeTok{x =} \StringTok{"Interval"}\NormalTok{, }\AttributeTok{y =} \StringTok{"Avg. Steps per day"}\NormalTok{)}
\end{Highlighting}
\end{Shaded}

\includegraphics{PA1_template_files/figure-latex/unnamed-chunk-6-1.pdf}

\begin{enumerate}
\def\labelenumi{\arabic{enumi}.}
\setcounter{enumi}{1}
\tightlist
\item
  ¿Qué intervalo de 5 minutos, de media en todos los días del conjunto
  de datos, contiene el máximo número de pasos?
\end{enumerate}

\begin{Shaded}
\begin{Highlighting}[]
\NormalTok{IntervalDT[steps }\SpecialCharTok{==} \FunctionTok{max}\NormalTok{(steps), .(}\AttributeTok{max\_interval =}\NormalTok{ interval)]}
\end{Highlighting}
\end{Shaded}

\begin{verbatim}
##    max_interval
## 1:          835
\end{verbatim}

\hypertarget{imputar-valores-perdidos}{%
\subsection{Imputar valores perdidos}\label{imputar-valores-perdidos}}

\begin{enumerate}
\def\labelenumi{\arabic{enumi}.}
\tightlist
\item
  Calcule y comunique el número total de valores perdidos en el conjunto
  de datos (i.e.~el número total de filas con 𝙽𝙰s)
\end{enumerate}

\begin{Shaded}
\begin{Highlighting}[]
\NormalTok{activityDT[}\FunctionTok{is.na}\NormalTok{(steps), .N ]}
\end{Highlighting}
\end{Shaded}

\begin{verbatim}
## [1] 2304
\end{verbatim}

\begin{Shaded}
\begin{Highlighting}[]
 \CommentTok{\# solución alternativa}

\FunctionTok{nrow}\NormalTok{(activityDT[}\FunctionTok{is.na}\NormalTok{(steps),])}
\end{Highlighting}
\end{Shaded}

\begin{verbatim}
## [1] 2304
\end{verbatim}

\begin{enumerate}
\def\labelenumi{\arabic{enumi}.}
\setcounter{enumi}{1}
\tightlist
\item
  Diseñe una estrategia para rellenar todos los valores que faltan en el
  conjunto de datos. La estrategia no tiene por qué ser sofisticada. Por
  ejemplo, puede utilizar la media/mediana de ese día, o la media de ese
  intervalo de 5 minutos, etc.
\end{enumerate}

\begin{Shaded}
\begin{Highlighting}[]
\CommentTok{\# Filling in missing values with median of dataset. }
\NormalTok{activityDT[}\FunctionTok{is.na}\NormalTok{(steps), }\StringTok{"steps"}\NormalTok{] }\OtherTok{\textless{}{-}}\NormalTok{ activityDT[, }\FunctionTok{c}\NormalTok{(}\FunctionTok{lapply}\NormalTok{(.SD, median, }\AttributeTok{na.rm =} \ConstantTok{TRUE}\NormalTok{)), .SDcols }\OtherTok{=} \FunctionTok{c}\NormalTok{(}\StringTok{"steps"}\NormalTok{)]}
\end{Highlighting}
\end{Shaded}

\begin{enumerate}
\def\labelenumi{\arabic{enumi}.}
\setcounter{enumi}{2}
\tightlist
\item
  Cree un nuevo conjunto de datos que sea igual al conjunto de datos
  original pero con los datos que faltan rellenados.
\end{enumerate}

\begin{Shaded}
\begin{Highlighting}[]
\NormalTok{data.table}\SpecialCharTok{::}\FunctionTok{fwrite}\NormalTok{(}\AttributeTok{x =}\NormalTok{ activityDT, }\AttributeTok{file =} \StringTok{"data/tidyData.csv"}\NormalTok{, }\AttributeTok{quote =} \ConstantTok{FALSE}\NormalTok{)}
\end{Highlighting}
\end{Shaded}

\begin{enumerate}
\def\labelenumi{\arabic{enumi}.}
\setcounter{enumi}{3}
\tightlist
\item
  Haz un histograma del número total de pasos dados cada día y calcula y
  comunica la media y la mediana del número total de pasos dados al día.
  ¿Difieren estos valores de las estimaciones de la primera parte de la
  tarea? ¿Cuál es el impacto de la imputación de los datos que faltan en
  las estimaciones del número total de pasos diarios?
\end{enumerate}

\begin{Shaded}
\begin{Highlighting}[]
\CommentTok{\# Número total de pasos dados al día}
\NormalTok{Total\_Steps }\OtherTok{\textless{}{-}}\NormalTok{ activityDT[, }\FunctionTok{c}\NormalTok{(}\FunctionTok{lapply}\NormalTok{(.SD, sum)), .SDcols }\OtherTok{=} \FunctionTok{c}\NormalTok{(}\StringTok{"steps"}\NormalTok{), by }\OtherTok{=}\NormalTok{ .(date)] }

\CommentTok{\#media y mediana del número total de pasos dados al día}
\NormalTok{Total\_Steps[, .(}\AttributeTok{Mean\_Steps =} \FunctionTok{mean}\NormalTok{(steps), }\AttributeTok{Median\_Steps =} \FunctionTok{median}\NormalTok{(steps))]}
\end{Highlighting}
\end{Shaded}

\begin{verbatim}
##    Mean_Steps Median_Steps
## 1:    9354.23        10395
\end{verbatim}

\begin{Shaded}
\begin{Highlighting}[]
\FunctionTok{ggplot}\NormalTok{(Total\_Steps, }\FunctionTok{aes}\NormalTok{(}\AttributeTok{x =}\NormalTok{ steps)) }\SpecialCharTok{+} \FunctionTok{geom\_histogram}\NormalTok{(}\AttributeTok{fill =} \StringTok{"blue"}\NormalTok{, }\AttributeTok{binwidth =} \DecValTok{1000}\NormalTok{) }\SpecialCharTok{+} \FunctionTok{labs}\NormalTok{(}\AttributeTok{title =} \StringTok{"Daily Steps"}\NormalTok{, }\AttributeTok{x =} \StringTok{"Steps"}\NormalTok{, }\AttributeTok{y =} \StringTok{"Frequency"}\NormalTok{)}
\end{Highlighting}
\end{Shaded}

\includegraphics{PA1_template_files/figure-latex/unnamed-chunk-11-1.pdf}

\begin{longtable}[]{@{}lll@{}}
\toprule
Tipo de estimacion & median\_Steps & Mediana\_Steps \\
\midrule
\endhead
primera Parte (con na) & 10765 & 10765 \\
Second Part (rellenar na con la mediana) & 9354.23 & 10395 \\
\bottomrule
\end{longtable}

\hypertarget{hay-diferencias-en-los-patrones-de-actividad-entre-los-duxedas-de-semana-y-los-fines-de-semana}{%
\subsection{¿Hay diferencias en los patrones de actividad entre los días
de semana y los fines de
semana?}\label{hay-diferencias-en-los-patrones-de-actividad-entre-los-duxedas-de-semana-y-los-fines-de-semana}}

1.Cree una nueva variable de factor en el conjunto de datos con dos
niveles: ``día de la semana'' y ``fin de semana'' que indique si una
fecha determinada es un día de la semana o del fin de semana.

\begin{Shaded}
\begin{Highlighting}[]
\CommentTok{\# JSolo hay que recrear la actividadDT desde cero y luego hacer el nuevo factor variable. (No es necesario, sólo quiero tener claro cuál es el proceso completo). }
\NormalTok{activityDT }\OtherTok{\textless{}{-}}\NormalTok{ data.table}\SpecialCharTok{::}\FunctionTok{fread}\NormalTok{(}\AttributeTok{input =} \StringTok{"data/activity.csv"}\NormalTok{)}
\NormalTok{activityDT[, date }\SpecialCharTok{:}\ErrorTok{=} \FunctionTok{as.POSIXct}\NormalTok{(date, }\AttributeTok{format =} \StringTok{"\%Y{-}\%m{-}\%d"}\NormalTok{)]}
\NormalTok{activityDT[, }\StringTok{\textasciigrave{}}\AttributeTok{Day of Week}\StringTok{\textasciigrave{}}\SpecialCharTok{:}\ErrorTok{=} \FunctionTok{weekdays}\NormalTok{(}\AttributeTok{x =}\NormalTok{ date)]}
\NormalTok{activityDT[}\FunctionTok{grepl}\NormalTok{(}\AttributeTok{pattern =} \StringTok{"Monday|Tuesday|Wednesday|Thursday|Friday"}\NormalTok{, }\AttributeTok{x =} \StringTok{"Day of Week"}\NormalTok{), }\StringTok{"weekday or weekend"}\NormalTok{] }\OtherTok{\textless{}{-}} \StringTok{"weekday"}
\NormalTok{activityDT[}\FunctionTok{grepl}\NormalTok{(}\AttributeTok{pattern =} \StringTok{"Saturday|Sunday"}\NormalTok{, }\AttributeTok{x =} \StringTok{"Day of Week"}\NormalTok{), }\StringTok{"weekday or weekend"}\NormalTok{] }\OtherTok{\textless{}{-}} \StringTok{"weekend"}
\NormalTok{activityDT[, }\StringTok{\textasciigrave{}}\AttributeTok{weekday or weekend}\StringTok{\textasciigrave{}} \SpecialCharTok{:}\ErrorTok{=} \FunctionTok{as.factor}\NormalTok{(}\StringTok{"weekday or weekend"}\NormalTok{)]}
\FunctionTok{head}\NormalTok{(activityDT, }\DecValTok{10}\NormalTok{)}
\end{Highlighting}
\end{Shaded}

\begin{verbatim}
##     steps       date interval Day of Week weekday or weekend
##  1:    NA 2012-10-01        0       lunes weekday or weekend
##  2:    NA 2012-10-01        5       lunes weekday or weekend
##  3:    NA 2012-10-01       10       lunes weekday or weekend
##  4:    NA 2012-10-01       15       lunes weekday or weekend
##  5:    NA 2012-10-01       20       lunes weekday or weekend
##  6:    NA 2012-10-01       25       lunes weekday or weekend
##  7:    NA 2012-10-01       30       lunes weekday or weekend
##  8:    NA 2012-10-01       35       lunes weekday or weekend
##  9:    NA 2012-10-01       40       lunes weekday or weekend
## 10:    NA 2012-10-01       45       lunes weekday or weekend
\end{verbatim}

\begin{enumerate}
\def\labelenumi{\arabic{enumi}.}
\setcounter{enumi}{1}
\tightlist
\item
  Haga un gráfico de panel que contenga un gráfico de serie temporal (es
  decir, 𝚝𝚢𝚙𝚎 = ``𝚕'') del intervalo de 5 minutos (eje x) y el número
  medio de pasos dados, promediado en todos los días de la semana o del
  fin de semana (eje y). Consulte el archivo README en el repositorio de
  GitHub para ver un ejemplo de cómo debería ser este gráfico utilizando
  datos simulados.
\end{enumerate}

\begin{Shaded}
\begin{Highlighting}[]
\NormalTok{activityDT[}\FunctionTok{is.na}\NormalTok{(steps), }\StringTok{"steps"}\NormalTok{] }\OtherTok{\textless{}{-}}\NormalTok{ activityDT[, }\FunctionTok{c}\NormalTok{(}\FunctionTok{lapply}\NormalTok{(.SD, median, }\AttributeTok{na.rm =} \ConstantTok{TRUE}\NormalTok{)), .SDcols }\OtherTok{=} \FunctionTok{c}\NormalTok{(}\StringTok{"steps"}\NormalTok{)]}
\NormalTok{IntervalDT }\OtherTok{\textless{}{-}}\NormalTok{ activityDT[, }\FunctionTok{c}\NormalTok{(}\FunctionTok{lapply}\NormalTok{(.SD, mean, }\AttributeTok{na.rm =} \ConstantTok{TRUE}\NormalTok{)), .SDcols }\OtherTok{=} \FunctionTok{c}\NormalTok{(}\StringTok{"steps"}\NormalTok{), by }\OtherTok{=}\NormalTok{ .(interval)]}

\FunctionTok{ggplot}\NormalTok{(IntervalDT , }\FunctionTok{aes}\NormalTok{(}\AttributeTok{x =}\NormalTok{ interval , }\AttributeTok{y =}\NormalTok{ steps, }\AttributeTok{color=}\StringTok{"weekday or weekend"}\NormalTok{)) }\SpecialCharTok{+} \FunctionTok{geom\_line}\NormalTok{() }\SpecialCharTok{+} \FunctionTok{labs}\NormalTok{(}\AttributeTok{title =} \StringTok{"Avg. Daily Steps by Weektype"}\NormalTok{, }\AttributeTok{x =} \StringTok{"Interval"}\NormalTok{, }\AttributeTok{y =} \StringTok{"No. of Steps"}\NormalTok{) }\SpecialCharTok{+} \FunctionTok{facet\_wrap}\NormalTok{(}\SpecialCharTok{\textasciitilde{}}\StringTok{\textquotesingle{}weekday or weekend\textquotesingle{}}\NormalTok{ , }\AttributeTok{ncol =} \DecValTok{1}\NormalTok{, }\AttributeTok{nrow=}\DecValTok{2}\NormalTok{)}
\end{Highlighting}
\end{Shaded}

\includegraphics{PA1_template_files/figure-latex/unnamed-chunk-13-1.pdf}

\end{document}
